\documentclass[]{article}
\usepackage{xeCJK}
\setCJKmainfont{WenQuanYi Zen Hei Sharp}
\usepackage{amsmath} %可以使用align
\usepackage[top=0.95in,bottom=0.95in,left=1.25in,right=1.2in]{geometry} %页边距

\begin{document}

\begin{titlepage}
\begin{center}
\LARGE 毕业设计理论笔记\\
[1.5cm]
\large{钟\ 威}
\vfill
\large 最后更改时间:\today \\
\end{center}
\end{titlepage}

\section*{凸规划问题基础}
\subsection*{$P_{17}$ 定义 1.2.1}
称S是凸集,如果对于任意的$x_1,x_2 \in S$和任意的$\lambda \in [0,1]$,都有:
$$ 
{\lambda}x_1 + (1 - {\lambda})x_2 \in S 
\eqno{(1.2.1)} $$

\subsection*{$P_{17}$ 定义 1.2.3}
任意的$ x_1,x_2 \in S $和任意的$ \lambda \in (0,1)$,都有:
$$ 
f({\lambda}x_1 + (1 - \lambda)x_2) \le {\lambda}f(x_1) + (1 - \lambda)f(x_2) 
\eqno{(1.2.2)} $$

\subsection*{$P_{19}$ 定理 1.2.4}
$f(x)$是凸函数的充要条件是:对于S中的任意一点$\bar{x}$,都有:
$$ 
f(x) \ge f(\bar{x}) + \nabla f(\bar{x})^T(x-\bar{x})
\eqno{(1.2.6)} $$

\subsection*{$P_{20}$ 定义 1.2.6}
\begin{align*}
\tag{1.2.10}
\min \qquad & f_0(x), \ x \in R^n \\
\tag{1.2.11}
s.t. \qquad & f_i(x) \le 0, \ i = 1, \ldots , m \\
\tag{1.2.12}
& h_i(x) = {a_i^T}x - b_i = 0, \  i = 1, \ldots, p. 
\end{align*}
\quad 
其中$f_0(x)$和$f_i(x)$都是定义在$R^n$上的连续可微凸函数,而$h_i(x)$是线性函数。

\subsection*{$P_{20}$ 引理 1.2.8}
若$f(x)$是$R^n$上的凸函数,则对于任意的$c \in R$,水平集:
$$ S = \{x|f(x) \le c, x \in R^n\}
\eqno{(1.2.16)} $$
是凸集。

\subsection*{$P_{20}$ 定理 1.2.10}
考虑凸规划问题(1.2.10)$\to$(1.2.12),若$x^*$是它的局部解,则$x^*$也是它的整体解。

\subsection*{$P_{21}$ 定理 1.2.12}
若凸规划问题(1.2.10)$\to$(1.2.12)中的目标函数$f_0(x)$是严格凸函数,则该问题的解唯一。

\subsection*{$P_{21}$ 定义 1.2.13}
设凸规划问题(1.2.10)$\to$(1.2.12)中变量$x$具有式:
$$ x = \left( \begin{array}{c} x_{1} \\
x_{2} \end{array} \right), 
\quad x_i \in R^{m_i}, 
\quad i = 1,2. 
\quad m_1 + m_2 = n.
\eqno{(1.2.22)} $$
所示的分划。称$x^*_1$是该问题关于$x_1$的解,如果存在$x^*_2 \in R^{m_2}$,使得$ x^* = ({x^*_1}^T, {x^*_2}^T)^T $是该问题的解。

\subsection*{$P_{22}$ 定理 1.2.15}
设凸规划问题(1.2.10)$\to$(1.2.12)中变量$x$具有式(1.2.22)的分划,记$f_0(x) = \mathbf{F}_0(x_1, x_2)$。若$\mathbf{F}_0(x_1, x_2)$分别是变量$x_1$和$x_2$的严格凸函数,则该问题对$x_1$的解唯一。

\section*{凸规划的对偶理论}

\subsection*{$P_{23}$ 定义 1.2.16 对偶问题}
引进Lagrange函数:
$$
\mathbf{L}(x, \lambda, \nu) = f_0(x) + \sum_{i = 1}^{m}{\lambda}_if_i(x) + \sum_{i = 1}^{p}{\nu}_ih_i(x)
\eqno{(1.2.35)} $$
其中$\lambda = ({\lambda}_1, \ldots, {\lambda}_m)^T$和$\nu = ({\nu}_1, \ldots, {\nu}_m)^T$是Lagrange乘子向量。
由书上$P_{23}$的论述可知,在$\lambda \le 0$时:$$\inf_{x \in R^n}\mathbf{L}(x, \lambda, \nu)$$
是$f_0(x)$的下界。
对于要找到最好的下界的问题,我们把:
\begin{align*}
\tag{1.2.40}
\max \qquad & g(\lambda, \nu) = \inf_{x \in R^n}\mathbf{L}(x, \lambda, \nu) \\
\tag{1.2.41}
s.t. \qquad & \lambda \le 0
\end{align*}

称为问题(1.2.10)$\to$(1.2.12)关于Lagrange函数(1.2.35)的对偶问题,或简称为问题(1.2.10)$\to$(1.2.12)的对偶问题。称(1.2.10)$\to$(1.2.12)为原始问题。

\subsection*{$P_{24}$ 定义 1.2.17 对偶间隙}
称原始问题的最优值与对偶问题的最优值之差为原始问题的对偶间隙。

\subsection*{$P_{24}$ 定理 1.2.20 Slater条件}
强对偶定理讨论的是对偶间隙为零的情况,我们需要“约束规格”来保证对偶间隙为零。对于凸规划问题(1.2.10)$\to$(1.2.12),最简单的约束规格是满足Slater条件:如果存在可行点$x$,使得:
$$
f_i(x) < 0, \ i=1, \ldots, m; \ {a_i^T}x - b_i = 0, \  i = 1, \ldots, p
\eqno{(1.2.44)} $$
特别地,当凸规划问题的前$k$个不等式约束为线性约束:$f_i(x) = {a_i^T}x - b_i \le 0, \ i = 1, \ldots, k$时,条件可进一步宽松为书上(1.2.45)$\to$(1.2.46)。

\subsection*{$P_{25}$ 定理 1.2.21 强对偶定理}
对于凸规划问题(1.2.10)$\to$(1.2.12),若它满足Slater条件,则他的对偶间隙为零。进一步,若还知原始问题的最优值可以达到,即存在最优解$x^*$,则对偶问题的最优值也可以达到,即存在对偶问题的整体解$({\lambda}^*, {\nu}^*)$使得:
\begin{align*}
f_0(x^*) &= \inf{f_0(x)} \\ 
&= \sup\{g({\lambda}, {\nu})|\lambda \le 0\} \\
&= \max\{g({\lambda}, {\nu})|\lambda \le 0\} \\
&= g({\lambda}^*, {\nu}^*)
\tag{1.2.47}
\end{align*}

\end{document}
